\documentclass[journal,12pt,twocolumn]{IEEEtran}
\usepackage{cite}
\usepackage{amsmath,amssymb,amsfonts,amsthm}
\usepackage{algorithmic}
\usepackage{graphicx}
\usepackage{textcomp}
\usepackage{xcolor}
\usepackage{txfonts}
\usepackage{listings}
\usepackage{enumitem}
\usepackage{mathtools}
\usepackage{gensymb}
\usepackage[breaklinks=true]{hyperref}
\usepackage{tkz-euclide} % loads  TikZ and tkz-base
\usepackage{listings}
\usepackage{float}


\begin{document}
\providecommand{\pr}[1]{\ensuremath{\Pr\left(#1\right)}}
\providecommand{\prt}[2]{\ensuremath{p_{#1}^{\left(#2\right)} }}        % own macro for this question
\providecommand{\qfunc}[1]{\ensuremath{Q\left(#1\right)}}
\providecommand{\sbrak}[1]{\ensuremath{{}\left[#1\right]}}
\providecommand{\lsbrak}[1]{\ensuremath{{}\left[#1\right.}}
\providecommand{\rsbrak}[1]{\ensuremath{{}\left.#1\right]}}
\providecommand{\brak}[1]{\ensuremath{\left(#1\right)}}
\providecommand{\lbrak}[1]{\ensuremath{\left(#1\right.}}
\providecommand{\rbrak}[1]{\ensuremath{\left.#1\right)}}
\providecommand{\cbrak}[1]{\ensuremath{\left\{#1\right\}}}
\providecommand{\lcbrak}[1]{\ensuremath{\left\{#1\right.}}
\providecommand{\rcbrak}[1]{\ensuremath{\left.#1\right\}}}
\newcommand{\sgn}{\mathop{\mathrm{sgn}}}
\providecommand{\abs}[1]{\left\vert#1\right\vert}
\providecommand{\res}[1]{\Res\displaylimits_{#1}} 
\providecommand{\norm}[1]{\left\lVert#1\right\rVert}
%\providecommand{\norm}[1]{\lVert#1\rVert}
\providecommand{\mtx}[1]{\mathbf{#1}}
\providecommand{\mean}[1]{E\left[ #1 \right]}
\providecommand{\cond}[2]{#1\middle|#2}
\providecommand{\fourier}{\overset{\mathcal{F}}{ \rightleftharpoons}}
\newenvironment{amatrix}[1]{%
  \left(\begin{array}{@{}*{#1}{c}|c@{}}
}{%
  \end{array}\right)
}
%\providecommand{\hilbert}{\overset{\mathcal{H}}{ \rightleftharpoons}}
%\providecommand{\system}{\overset{\mathcal{H}}{ \longleftrightarrow}}
	%\newcommand{\solution}[2]{\textbf{Solution:}{#1}}
\newcommand{\solution}{\noindent \textbf{Solution: }}
\newcommand{\cosec}{\,\text{cosec}\,}
\providecommand{\dec}[2]{\ensuremath{\overset{#1}{\underset{#2}{\gtrless}}}}
\newcommand{\myvec}[1]{\ensuremath{\begin{pmatrix}#1\end{pmatrix}}}
\newcommand{\mydet}[1]{\ensuremath{\begin{vmatrix}#1\end{vmatrix}}}
\newcommand{\myaugvec}[2]{\ensuremath{\begin{amatrix}{#1}#2\end{amatrix}}}
\providecommand{\rank}{\text{rank}}
\providecommand{\pr}[1]{\ensuremath{\Pr\left(#1\right)}}
\providecommand{\qfunc}[1]{\ensuremath{Q\left(#1\right)}}
	\newcommand*{\permcomb}[4][0mu]{{{}^{#3}\mkern#1#2_{#4}}}
\newcommand*{\perm}[1][-3mu]{\permcomb[#1]{P}}
\newcommand*{\comb}[1][-1mu]{\permcomb[#1]{C}}
\providecommand{\qfunc}[1]{\ensuremath{Q\left(#1\right)}}
\providecommand{\gauss}[2]{\mathcal{N}\ensuremath{\left(#1,#2\right)}}
\providecommand{\diff}[2]{\ensuremath{\frac{d{#1}}{d{#2}}}}
\providecommand{\myceil}[1]{\left \lceil #1 \right \rceil }
\newcommand\figref{Fig.~\ref}
\newcommand\tabref{Table~\ref}
\newcommand{\sinc}{\,\text{sinc}\,}
\newcommand{\rect}{\,\text{rect}\,}
%%
%	%\newcommand{\solution}[2]{\textbf{Solution:}{#1}}
%\newcommand{\solution}{\noindent \textbf{Solution: }}
%\newcommand{\cosec}{\,\text{cosec}\,}
%\numberwithin{equation}{section}
%\numberwithin{equation}{subsection}
%\numberwithin{problem}{section}
%\numberwithin{definition}{section}
%\makeatletter
%\@addtoreset{figure}{problem}
%\makeatother

%\let\StandardTheFigure\thefigure
\bibliographystyle{IEEEtran}


\vspace{3cm}

\title{
Ncert exempler
}
\author{ KUNWAR DUSHYANT SINGH EE22BTECH11031}


\maketitle

\newpage

\maketitle
\textbf{Question 12.13.3.38}\\
A and B throw a pair of dice alternately. A wins the game if he gets a total of
6 and B wins if she gets a total of 7. It A starts the game, find the probability of
winning the game by A in third throw of the pair of dice.\\
\solution
Let state defined be
\begin{table}[H]
\begin{tabular}{|c|c|}
\hline
State &description \\ \hline
 $S_0$ & A rolls dice\\ \hline
 $S_1$ & B rolls dice\\\hline
 $S_2$ & A wins\\\hline
 $S_3$ & B wins\\\hline
 $S_4$  & game stops\\\hline
\end{tabular}
\caption{States}
\label{tab:exemplar/12/13/3/38}
\end{table}
Markov chain\\
\begin{figure}[ht!]
    \centering
    \resizebox{\linewidth}{!}{\input{fig.tex}}
    \caption{State diagram generated using LatexTikZ}
    \label{fig:Statediagramdiecoin}
\end{figure}
\\
Initial state vector\\
\begin{align}
S &= \myvec{ 1\\ 0\\0\\0\\0\\}
\end{align}
Transition matrix\\
\begin{align}
P &= \myvec{
    0 & \frac{31}{36} & \frac{5}{36} & 0 & 0 \\
    \frac{5}{6} & 0 & 0 & \frac{1}{6} & 0 \\
    0 & 0 & 0 & 0 & 1 \\
    0 & 0 & 0 & 0 & 1 \\
    0 & 0 & 0 & 0 & 1
}
\end{align}
Probablity of A winning in third throw given A starts first\\
\begin{align}
S_3 &= S_0^{\top}P^3\\
&= \myvec{ 1\\ 0\\0\\0\\0\\}^{\top}\myvec{
     & \frac{4805}{7776} & \frac{775}{7776} & 0 & \frac{61}{216} \\
     & \frac{775}{1296} & 0 & 0 & \frac{155}{1296} & \frac{61}{216} \\
     & 0 & 0 & 0 & 0 & 1 \\
     & 0 & 0 & 0 & 0 & 1 \\
     & 0 & 0 & 0 & 0 & 1
}\\
S_3 &= \myvec{0\\ 0.617\\0.099\\0}\\
S_3[2] &= 0.099
\end{align}
\end{document}

